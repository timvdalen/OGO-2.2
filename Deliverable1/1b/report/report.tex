\documentclass[12pt]{article}
\usepackage{ltcadiz}
\usepackage{listings}
\lstset{breaklines=true, numbers=left}

\title{Report on OGO 2.2 Softwarespecification assignment 1b}
\author{
        Tim van Dalen, Tony Nan and Ferry Timmers\\ OGO 2.2 group 6 \\
                Department of Computer Science\\
        Technical University Eindhoven\\
}
\date{\today}

\begin{document}

\maketitle

\begin{abstract}
This document presents a formal specification and implementation of assignment 1b of OGO 2.2 Softwarespecification. Also included are the informal requirement of the stakeholder, their judgement about the formal specification and testcases for the implementation.
\end{abstract}

\section{Informal requirements}
\paragraph{Input} Provided is a simple (ASCII) file with a few numbers separated by a space of arbitrary size. There is no limit to the size of the file.

\paragraph{Output} Take all unique numbers from the file of the form $n^3$ with $n$ a natural number and select the lower median of these numbers. A number is of the form $n^3$, when the cube'd root of the number is a natural number.

We can find the lower median of a sequence of numbers by sorting the sequence and selecting the number in the middle of the sequence. If the size of the sequence is even, the lower median is the left element of the two elements in the middle.

\section{Formal specification}

This section describes a formal specification adhering to the informal requirements described above.

We first have to define an enumeration \textit{Cubes} that contains all sets of natural numbers for which each element is a cube of a natural number.
\begin{axdef}
Cubes == \{ x : \power \nat | (\forall y : x | (\exists z : \nat @ y^3 = z) )\}
\end{axdef}

We define a function that calculates the lower median of a set of natural numbers. It defines two disjoint subsets $a$ and $b$ that are the two (almost) equal halves of the input. All elements in $a$ are smaller than $b$, and the output is the number that is wedged between the two halves. The union of $a$, $b$ and the output is the whole input.
\begin{schema}{Median}
s? : \power \nat \\
m! : \nat
\where
\exists a,b : \power s? | [ \: a \union b \union m! = s? \\
                          \t1 a \cap b = \emptyset \\
                          \t1 m! \notin ( a \union b ) \\
                          \t1 \# b - \# a = 0 \: \vee \# b - \# a = 1 \\
                          \t1 \forall x : a , y : b | x < m! < y \: ]
\end{schema}

Now we define the input and output of our specification. The output is computed by taking a set from \textsl{Cubes} with only cubes from the input (and no other cubes) and computing the median from it.
\begin{axdef}
Input : \seq \nat
\end{axdef}

\begin{axdef}
Output : \nat
\where
\exists X : Cubes | [ \: X \subseteq \ran Input \\ \t1 \forall Y : \ran Input \setminus X | Y \not\subseteq Cubes \\ \t1 Output = Median (X) \: ]
\end{axdef}

\section{Judgement of the formal specification}

\section{Testcases}

\paragraph{Null testcase}
This case checks whether an empty input creates the correct output. \\
Input: \{\} \\
Expected output: 0 (test succeeded).

\paragraph{Double one testcase}
This case checks if the output is still correct when the input is 5 times the same number. \\
Input: \{1 1 1 1 1\} \\
Expected output: 1 (test succeeded)

\paragraph{Inverse order}
This case checks whether the algorithm works if the order is inversed. \\
Input: \{1000 512 216 64 8 0\} \\
Expected output: 4 (test succeeded)

\paragraph{Invalid input}
This case checks if an exception is given when the input is incorrect. \\
Input \{1000 512 217 64 8 0\} \\
Expected output: IllegalArgumentException: incorrect input: $6^3$ != 217 (test succeeded)

\paragraph{Combination}
In this test case there are double values and they aren't ordered.
Input: \{27 8 216 8 729 729 216 1 8 1 216\} \\
Expected output: 3 (test succeeded)

\section{Implementation}
Over here we, again, chose java as our programming language because that's the language that everyone in our group knows. The implementation follows our specification correctly, but returns a zero if no input is given.

First, in line 21 the input's cubed root is calculated, then we check in lines 23-26 if that root is a natural number by throwing an exception that if the cubed result differs from the input. If no exception is thrown the result is added to an arrayList $n$ which is then sorted.

In lines 33-39 we remove double values to make sure that we only pick the unique numbers from the input.

In lines 41-45 we pick out the lower median or pick out 0 if the arrayList is empty.

All our testcases resulted in a succes and we've used the following imports: \textsl{java.io.InputStream}, \textsl{java.util.ArrayList}, \textsl{java.util.Collections} and \textsl{java.util.Scanner}.

\appendix

\section{Java implementation listing}
\begin{lstlisting}[language=java]
/**
 * This method first calculates the cubed root of all the natural numbers
 * on the input, then it puts all of them in an arraylist, sorts the
 * arraylist, removes non-unique values and picks out the lower median, this
 * lower median is returned.
 * @author Sander Breukink
 * @since  8-2-2012
 * @param  stream The input stream with integers disjunct bij spaces
 * @return The lower mean from the cubed roots of the unique input integers
 */
public static int FindLowMed(InputStream stream){
    Scanner scanner = new Scanner (stream);
    ArrayList<Integer> n = new ArrayList<Integer>();
    int lowerMedian;
    int input;
    int result;

    while (scanner.hasNextInt()){
        input = scanner.nextInt();
        //round for approaching the root (for example 1000^(1/3) = 9,99999999999 -> 10)
        result = (int) Math.round(Math.pow(input,1.0/3));

        if(Math.pow(result,3) != input){
            //check if the input was actually correct
            throw new IllegalArgumentException("incorrect input: "+result+"^3 != " + input);
        }
        //put all values of n in the arraylist
        n.add(result);
    }
    //sort the arraylist
    Collections.sort(n);

    for(int i = 0; i < n.size() - 1; i++){
        //remove all non-unique values
        if(n.get(i) == n.get(i+1)){
            n.remove(i+1);
            i--;
        }
    }

    if(n.size() > 0){
        lowerMedian = n.get((int) (n.size()-1)/2);
    }else{
        lowerMedian = 0;
    }

    return lowerMedian;
}
\end{lstlisting}

\end{document} 