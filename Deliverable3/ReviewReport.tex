\documentclass[a4paper,11pt]{article}
\usepackage{graphicx,listings,a4wide,dsfont}
%\usepackage[firstpage]{draftwatermark}
%\SetWatermarkLightness{0.5}
%\SetWatermarkScale{4}
\setcounter{tocdepth}{2}

\newcommand{\question}[2]{\medskip\par\noindent\textbf{#1}\\\hangindent=0.5cm#2}

\title{Report on OGO 2.2 \\ Software specification\\ Review formal specification group 3}
\author{
        Tim van Dalen, Tony Nan, Ferry Timmers, \\ Lasse Blaauwbroek, Femke Jansen, \\Jeroen Peters and Sander Breukink\\ OGO 2.2 group 6 \\
                Department of Computer Science\\
        Technical University Eindhoven\\
}
\date{\today}

\begin{document}

\maketitle

\begin{abstract}
This document contains the review of the formal specification of group 3. We first provide some general remarks about the overall structure of the formal specification. In the sections hereafter, we provide our remarks about the Z-specification, the state diagram and the MSC's; here, we distinguish between general remarks and individual remarks about specific parts of the specification. We provide our remarks as lists, to keep them organized and readable. Finally, we give our judgement of the formal description and grade the formal description.
\end{abstract}

\newpage
	
	\tableofcontents
	\newpage
	
	\section{General remarks}
	%During the development of the product, there was some communication with the stateholder.

\begin{description}
	\item[If a robot that does multiple steps at once crosses a special tile, what happens?] Hint tile: nothing. Home time: the robot wins. Conveyor belt: it gets dragged along.
	\item[If a robot tries to move to a tile that is occupied, can he move again directly after that?] There are no turns so this doesn't matter.
	\item[If a robot moves to a tile that is part of a conveyor belt (not the beginning and not the end), what happens?] The robot gets dragged along.
	\item[If a robot is 'trapped' by three broken robots after it steps of a conveyor belt, can it move back?] This can't happen by definition, all robots must be able to reach their home tiles. Conveyor belts are one way.
	\item[How should conveyor belt transportation work? In one move to the end or a certain amount of tiles per time unit?] In one time to the end.
	\item[What happens if a broken robot is swapped with a tile in the middle of a conveyor belt?] The tile before the broken robot is the new end tile and the tile after the broken robot becomes the start tile for the conveyor belt after that.
	\item[Can a broken robot block a hint or home tile?] No, a broken robot should be a tile type, like home or hint.
	\item[Can a conveyor belt be a rotation and a move belt at the same time?] Yes, it always is.
\end{description}

Both robots have the same functionality seeing as there are no turns. Robot A can do three times a move in the same time that robot B can do one three tile move. Because robot B does not jump but actually crosses the tiles, the result is the same, except for when they cross their home tile. The stakeholder decided that we should just implement it the way it's specified now.

On Tuesday the 28th of Februari, we had a meeting with the stakeholders about the role of the controller. We assumed that the controller actually had to control something, but the stakeholder did not fully agree with us. In our MSCs the controller communicated with all other parts of the game (the players, the view and the board) and, for instance, first asked the board for two tiles that could be switched and then asked it to switch those. In the opninion of the stakeholder, the controller should merely serve as a communications tunnel and not make any decisions itself. Eventually, we reached a compromise. The controller still requests everything from the other parts, but all requests have been made automic, i.e. the tile request functions are now one function that the board handles by itself.

    We were not content with the following things in the overall report:
    \begin{itemize}
        \item The order of sections is not logical: the Z specification should be after the MSC's and the Statecharts.
        \item The abstract is missing.
        \item There is no content page.
        \item The introduction is short and does not tell what is coming next, so it isn't what an introduction should be like. For example, the state diagrams and MSC's are not mentioned.
        \item There are no any use case scenario's (and also no matching state diagrams and MSC's).
        \item The design decisions that have been made aren not documented.
        \item The class diagram is missing.
        \item It is not specified how a game ends and how the individual components of the game (Controller, Board, etc.) can conclude and react to this.
    \end{itemize}
	
	\section{Remarks about the Z-specification}
    In this section, we organize our individual comments according to the order of the Z-schemas in the formal specification.
    \subsection{Overall remarks Z-specification}
    \begin{itemize}
        \item The description of your coordinate-system is very nice and understandable.
        \item Splitting the "MoveRequest" in Controller in several pieces was a wise decision; this way, you can keep things organized and introduce the reader step-by-step to the whole concept.
        \item Boolean is a pre-defined set in Z. The correct notation for the set of booleans is $\mathds{B}$.
        \item In sections 2 to  3.5, input-/output-variables are prefixed with a ?/!. In the remaining sections of the Z-specification, the input-/output-variables are postfixed with a ?/!. This is inconsistent and partly incorrect; input/output-variables should be postfixed with a ?/!.
        \item Some variables are unbounded or even undefined. For example, in the schema DoKill, a variable $s$ is used, but it is not clear where this variable comes from and what it's type is. The same holds for the $s?$-variable in DoMove.
        \item The MoveRequest in player is actually not part of the player-entity/class. It is requested by a player via the Controller. The MoveRequest in player also checks whether is move is possible, but that is not the job of the player; the board should check whether a move request is valid.
        \item The notion of "safe tiles" is not reflected in the Z-specification. This is an important factor when a fox wants to kill a dolphin and vice versa.
        \item The notion of turns is explained informally, but not reflected formally in the Z-specification. For example, the tides occur once each player has no moves left.
        \item The invariant "Two adjacent tiles (meaning they have a common edge) may only differ in two units of elevation" is not maintained in the Z-specification.
        \item You do specify a Number of Moves schema, but do not use it in MoveRequest or DoMove. A move can only be requested (and executed) if a player has any moves left.
        \item The board has a size that depends on the number of pieces. This is not reflected in the formal specification.
        \item It is not always clear where the input-, output- and dummy-variables are used for in the Z-schemas. Additional explanation would be appreciated.
    \end{itemize}

    \subsection{Pieces and players}
    \begin{itemize}
        \item One player has all pieces of type 'fox' and one player has all pieces of type 'dolphin'. You only specify that each player has pieces of the same type.
    \end{itemize}

    \subsection{Board}
    \begin{itemize}
        \item In the informal explanation of Flood, you state you call the Flood-function twice; however, we do not see how this is reflected in the Z-schema.
        \item In OccupiedBySameAnimal, the Z-schema is correct, but the informal explanation is not. You state the following: if the destination tile of a move request is occupied by an animal of the same type, then isOccupied is true and the move to this tile is possible. In this case, the move is, of course, not possible.
        \item In MakeMove, a conjunction is missing between the parts of the fox and the dolphin.
        \item In ShortcutPossible, both the dolphins and foxes use the bridges now as a shortcut. Dolphins should, of course, use the caves.
        \item In Init, two foxes and two dolphins should not be placed on the same tile. This is captured in MoveRequest, but it must also be captured in the initial configuration.
        \item In the informal explanation of getNrMoves, you say that the operation calculates the number of moves "automagically".
    \end{itemize}

    \subsection{Viewer and Controller}
    \begin{itemize}
        \item The initial number of foxes and dolphins can be predefined by the players. It is not clear whether this is the $n?$-variable in init; if it is, explain this in the informal description.
        \item In Init, all foxes should be placed on the land and all dolphins in the water.
    \end{itemize}

	\section{Remarks about the State Diagram}
	%\subsection{Class diagram}
\includegraphics[width=\linewidth]{classdiagram.pdf}

    ToDo
P
    \section{Remarks about Message Sequence Chart}
	%\subsection{Class diagram}
\includegraphics[width=\linewidth]{classdiagram.pdf}

    ToDo

    \section{Judgement and grading}
    \subsection{Consistency}
    ToDo

    \subsection{Correspondence to the informal description}
    ToDo

    \subsection{Completeness}
    ToDo

    \subsection{Explanation and coherence}
    ToDo
\end{document} 