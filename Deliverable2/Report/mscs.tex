This section contains the message sequence charts, both of high and class level.
\subsection{High Level Message Sequence Chart}
	The graph in figure~\ref{fig:msc:high} represents our high level message sequence chart and shows how a normal program flow is modeled by MSCs. The graph consists of two parts that run concurrently, the viewer part and the main game part. The viewer part will make sure the viewer updates regularly. The main game part follows the flow of a normal game.
	
	\begin{figure}[h]
		\digraph[scale=0.5]{HMSC}{
begin2 [label="",shape="invtriangle"];
end2 [label="",shape="triangle"];
initview [label="Initialize viewer"];
updateview [label="Update viewer"];
begin2->initview;
initview->updateview;
updateview->updateview;
updateview->end2;
begin [label="",shape="invtriangle"];
end [label="",shape="triangle"];
p1 [label="",shape="point"];
p2 [label="",shape="point"];
p3 [label="",shape="point"];
p4 [label="",shape="point"];
p5 [label="",shape="point"];
p6 [label="",shape="point"];
p7 [label="",shape="point"];
p8 [label="",shape="point"];
init [label="Initialize",shape="Mrecord"];
mvreq [label="Robot move request",shape="Mrecord"];
mvrej [label="Reject move",shape="Mrecord"];
retnt [label="Return Normal tile",shape="Mrecord"];
retht [label="Return Hint tile",shape="Mrecord"];
retct [label="Return Conveyor tile",shape="Mrecord"];
retmt [label="Return Home tile",shape="Mrecord"];
ordex [label="Ordinary exchange",shape="Mrecord"];
spcex [label="Special exchange",shape="Mrecord"];
endgame [label="End game",shape="Mrecord"];
begin->p1;
p1->init;
init->p2;
p2->mvreq;
mvreq->p3;
p3->mvrej;
mvrej->p2;
p3->p4;
p4->retnt;
p4->retht;
p4->retct;
p4->retmt;
retnt->p5;
retht->p5;
retct->p5;
p5->p6;
p6->ordex;
p6->spcex;
ordex->p7;
spcex->p7;
p7->p2;
retmt->endgame;
endgame->p8;
p8->p1;
p8->end;
}

		\label{fig:msc:high}
		\caption{The high level message sequence chart}
	\end{figure}

\subsection{Message Sequence Charts}
	This section contains the message sequence charts for our use cases. Below every MSC is the location of the MSC in the High Level Message Sequence Chart, which shows the possible routes that can be taken from the corresponding MSC. For example, below the MSC 'Initialize', both 'Initialize' and 'Move Request' are shown, since 'Move Request' is the next step after 'Initialize' has been completed. If several options for next moves exist, they are given in the same way as in the HMSC below; in the corresponding separate MSC is explained when and why a certain route is taken.

    A move request that results in a robot being moved to a certain tile is referred to as an 'Return ... tile'-MSC. This convention is used, because the board provides several responses, messages and signals based upon the outcome of the move request. For example, in 'Return Home tile', the board uses a \emph{WIN} response to indicate that a certain robot has reached its home tile. Also, in 'Return Hint tile', the board acknowledges the move request with a \emph{SUCCESS} and will thereafter send a separate message that contains the requested hint.

	The world entity is not a real part of our MSCs, but rather a link to the outside world. When an internal action ends in '()' it's a function call to a private function of the entity. Otherwise, it's an action within the called function.

	The following MSC syntax is used in this section:

	\begin{msc}
	msc{
        	a, b;

        	a -> b [label="A signal"];
        	a => b [label="A function call"];
        	b >> a [label="A return value"];
        	b rbox b [label="An internal action"];

	}
\end{msc}


	In several MSCs, we use co-regions to denote that the messages in these regions can be send in any order. For example, in the MSC 'Return Normal Tile', the order in which the board response and the notify to the Viewer are send is not important; it is more important that these messages or signals are received by the appropriate entities. As another example, consider the MSC 'Initialize'. In this MSC, it is important that several entities are initialized before others, like the board is initialized before the robots. Hence, in this MSC the order plays a central role for the flow of the programm. In our MSCs, the symbol $\vdots$ denotes the start and end of a co-region.

	\subsubsection{Initialize viewer}
	\begin{minipage}{\linewidth}
		The viewer is initialized by the world. Analogous to the HMSC, this happens concurrently to the Initialize MSC. \\
For this MSC, the assumption is made that there already exists a viewer and that it knows about the other components in the game, when the call from the outside world is received. After the call has been received, the viewer will initialize itself and will try to attach itself to the controller. Since initializing the viewer happens concurrently to initializing the other components of the game, the controller may not have been initialized yet; the viewer will continuously try to attach itself, until it receives a respond from the controller.
	
		\begin{msc}
msc
{

w [label="World"],
b [label="Board"],
c [label="Controller"],
v [label="Viewer"];

v box v [label=""],
b box b [label=""],
c box c [label=""],
w box w [label=""];

|||;

w => v [label="initialize(controller)"];
v rbox v [label="initialize"];
v => c [label="addViewer(viewer)"];
c >> v [label="self"];

...;
... [label="board waits a predefined time"];
...;

b rbox b [label="make snapshot"];
b -> c [label="notifyViewer(snapshot)"];
c -> v [label="notifyStateChange(snapshot)"];
v rbox v [label="updateView()"];

|||;

v box v [label="", textbgcolor="black"],
b box b [label="", textbgcolor="black"],
c box c [label="", textbgcolor="black"],
w box w [label="", textbgcolor="black"];

}
\end{msc}
\digraph[scale=0.5]{HMSC_init_view}{rankdir=LR;
p2 [label="",shape="point"];
init [label="Initialize",shape="Mrecord"];
initview [label="Initialize viewer",shape="Mrecord",style=filled];
mvreq [label="Robot move request",shape="Mrecord"];
init->initview;
initview->p2;
p2->mvreq
}

	\end{minipage}

    	\subsubsection{Update viewer}
	\begin{minipage}{\linewidth}
		During the game execution, the viewer updates its view whenever it gets notified by the controller that the board has changed. Note that this notification is not shown in the MSC below, because it is part of the 'Return ... tile' MSCs and the tiles exchange MSCs. In each of these MSCs, the board will always make a snapshot whenever the board has changed and send that snapshot to the viewer, via the controller.

		\begin{msc}
msc {

b [label="Board"],
c [label="Controller"],
d [label="Viewer"];

b box b [label=""],
d box d [label=""],
c box c [label=""];

|||;

b rbox b [label="make snapshot"];
b -> c [label="notifyViewer(snapshot)"];
c -> d [label="notifyStateChange(snapshot)"];
d rbox d [label="updateView()"];

|||;

d box d [label="", textbgcolor="black"],
c box c [label="", textbgcolor="black"],
b box b [label="", textbgcolor="black"];

}
\end{msc}
\digraph[scale=0.5]{HMSC_upview1}{
rankdir=LR;
p5 [label="Update viewer",shape="Mrecord",style=filled];
p6 [label="",shape="point"];
notrob1 [label="Notify robots",shape="Mrecord"];
p55 [label="",shape="point"];
p66 [label="",shape="point"];
retnt [label="Return Normal tile",shape="Mrecord"];
retct [label="Return Conveyor tile",shape="Mrecord"];
retht [label="Return Hint tile",shape="Mrecord"];
ordex [label="Ordinary exchange",shape="Mrecord"];
spcex [label="Special exchange",shape="Mrecord"];
retnt->p5;
retct->p5;
retht->p5;
p5->p55;
p55->p66;
p66->p6;
p55->notrob1;
notrob1->p66;
p6->ordex;
p6->spcex;
}
\digraph[scale=0.5]{HMSC_upview2}{
rankdir=LR;
begin [label="",shape="invtriangle"];
end [label="",shape="triangle"];
p1 [label="",shape="point"];
p8 [label="",shape="point"];
uv [label="Update viewer",shape="Mrecord",style=filled];
init [label="Initialize",shape="Mrecord"];
retmt [label="Return Home tile",shape="Mrecord"];
endgame [label="End game",shape="Mrecord"];
begin->p1;
p1->init;
retmt->uv->endgame;
endgame->p8;
p8->p1;
p8->end;
}
\digraph[scale=0.5]{HMSC_upview3}{
rankdir=LR;
p2 [label="",shape="point"];
p22 [label="",shape="point"];
p77 [label="",shape="point"];
mvreq [label="Robot move request",shape="Mrecord"];
ordex [label="Ordinary exchange",shape="Mrecord"];
spcex [label="Special exchange",shape="Mrecord"];
p7 [label="Update viewer",shape="Mrecord",style=filled];
notrob2 [label="Notify robots",shape="Mrecord"];
ordex->p7;
spcex->p7;
p7->p77;
p77->p22;
p22->p2;
p2->mvreq;
p77->notrob2;
notrob2->p22;
}
	\end{minipage}

	\subsubsection{Initialize}
	\begin{minipage}{\linewidth}
		When the game starts, the board is initialized. When this is done, the board sends a preInitialize to the controller. When the controller is done with that, the board will initialize all robots (choosing the appropriate rule for them), which will reply with an 'OK' when done. When all robots have been initialized, the board sends a postInitialize to the controller. All entities are now initialized. \\
	We introduced a separate pre- and post-initialize method, because the controller can not fully initialize without the robots and the robots, in turn, must be initialized with the controller. So, the pre-initialize is used to create the controller and the post-initialize is used to store all the robots that have been initialized by the board.

		\begin{msc}
msc
{

r1 [label="Robot1"],
r2 [label="Robot2"],
b [label="Controller"],
c [label="Board"];

r1 box r1 [label=""],
r2 box r2 [label=""],
b box b [label=""],
c box c [label=""];

|||;

c rbox c [label="initialize"];
c => b [label="preInitialize"];
b rbox b [label="preInitialize"];
b => c [label="done"];
c => r1 [label="initialize"]; 
r1 rbox r1 [label="initialize"];
c => r2 [label="initialize"]; 
r2 rbox r2 [label="initialize"];
r1 note r2 [label="All robots receive an initialize from the board."];
r1 >> c [label="done"];
r2 >> c [label="done"];
c => b [label="initialize(robotList)"];
b rbox b [label="initialize"];
c >> b [label="done"];

|||;

a box a [label="", textbgcolor="black"],
b box b [label="", textbgcolor="black"],
c box c [label="", textbgcolor="black"];

}
\end{msc}

    \end{minipage}

	\subsubsection{Robot move request}
	\begin{minipage}{\linewidth}
		A robot can make a move request with the controller, which will forward that to the board. The board will check the validity of the move itself with the robots rule and then check the validity of the move at the current state of the board. The internal actions 'get possible moves' and 'get possible rotations' are thus used by the board to check whether the move request is conform with the specified moves and rotations in the rule of the robot. Based upon the outcome of the move request, one of the five MSCs given in the HMSC below is executed. \\
Note that the move requests describe a move request in terms of local coordinates, since a robot does not know its exact location on the board. Also, the move request contains the robot that requested the move, along with the requested rotation.

		\begin{msc}
msc
{

d [label="Board"],
c [label="Controller"],
a [label="Robot"],
r [label="Rule"];

d box d [label=""],
c box c [label=""],
a box a [label=""],
r box r [label=""];

|||;

a => c [label="moveRequest(localCoords, robot)"];
c => d [label="moveRequest(localCoords, robot)"];
d => r [label="getPossibleMoves()"];
d << r [label="list of possible moves"];
d => r [label="getPossibleRotations()"];
d << r [label="list of possible rotations"];
d box d [label="canPlaceRobot(localCoords, robot)"];

|||;

a box a [label="", textbgcolor="black"],
c box c [label="", textbgcolor="black"],
d box d [label="", textbgcolor="black"],
r box r [label="", textbgcolor="black"];

}
\end{msc}
\digraph[scale=0.5]{HMSC_req}{
rankdir=LR;
p2 [label="",shape="point"];
p3 [label="",shape="point"];
p4 [label="",shape="point"];
mvreq [label="Robot move request",shape="Mrecord",style=filled];
mvrej [label="Reject move",shape="Mrecord"];
retnt [label="Return Normal tile",shape="Mrecord"];
retht [label="Return Hint tile",shape="Mrecord"];
retct [label="Return Conveyor tile",shape="Mrecord"];
retmt [label="Return Home tile",shape="Mrecord"];
init->p2;
p2->mvreq;
mvreq->p3;
p3->mvrej;
mvrej->p2;
p3->p4;
p4->retnt;
p4->retht;
p4->retct;
p4->retmt;
}

    \end{minipage}

	\subsubsection{Return Normal tile}
	\begin{minipage}{\linewidth}
		If the move request is valid, and the robot is moved to a normal tile, the board will return \emph{SUCCESS} to the controller, which will forward this message to the robot that was moved. The board also makes a snapshot and notifies the controller that the board has changed, which forwards this snapshot to the viewer. Internally, the board calculates the new location of the robot and save the location if the move request is valid. Note that the first internal actions of the board involve calculating and saving the new location of the moved robot, to keep the board up-to-date.

		\begin{msc}
msc
{

d [label="Board"],
c [label="Controller"],
a [label="Robot"],
v [label="Viewer"];

d box d [label=""],
c box c [label=""],
a box a [label=""],
v box v [label=""];

|||;

d rbox d [label="calculateNewLocation(localCoords, robot)"];
d rbox d [label="saveLocation(absCoords, robot)"];
d >> c [label="SUCCESS"];
c >> a [label="SUCCESS"];
c => v [label="notifyStateChange()"];

|||;

a box a [label="", textbgcolor="black"],
c box c [label="", textbgcolor="black"],
d box d [label="", textbgcolor="black"];

}
\end{msc}
\digraph[scale=0.5]{HMSC_mvnt}{
rankdir=LR;
p3 [label="",shape="point"];
p4 [label="",shape="point"];
p5 [label="",shape="point"];
p6 [label="",shape="point"];
mvreq [label="Robot move request",shape="Mrecord"];
mvrej [label="Reject move",shape="Mrecord"];
retnt [label="Return Normal tile",shape="Mrecord",style=filled];
ordex [label="Ordinary exchange",shape="Mrecord"];
spcex [label="Special exchange",shape="Mrecord"];
mvreq->p3;
p3->mvrej;
p3->p4;
p4->retnt;
retnt->p5;
p5->p6;
p6->ordex;
p6->spcex;
}

	\end{minipage}

	\subsubsection{Return Hint tile}
	\begin{minipage}{\linewidth}
		If the move request is valid, and the robot is moved to a hint tile, the board will first respond with \emph{SUCCESS}, to indicate that the move request has been granted. Hereafter, the board generates a hint then notifies the controller that the robot that moved should receive a hint. The controller will notify the viewer that the state of the board has changed and forward the hint to the robot. Note that the notifyHint from the board to the controller also contains a robot, since the controller must be able to identify the robot that should receive the hint. Of course, the viewer has to be notified again that the board has changed.

		\begin{msc}
msc
{

d [label="Board"],
c [label="Controller"],
a [label="Robot"],
v [label="Viewer"];

d box d [label=""],
c box c [label=""],
a box a [label=""],
v box v [label=""];

|||;

d rbox d [label="calculateNewLocation(localCoords, robot)"];
d rbox d [label="saveLocation(absCoords, robot)"];
d >> c [label="HINT"];
c => d [label="getHint(robot)"];
d rbox d [label="generate hint"];
d >> c [label="hint"];
c => a [label="notifyHint(hint)"];


|||;

a box a [label="", textbgcolor="black"],
c box c [label="", textbgcolor="black"],
d box d [label="", textbgcolor="black"];

}
\end{msc}
\digraph[scale=0.5]{HMSC_mvht}{
rankdir=LR;
p3 [label="",shape="point"];
p4 [label="",shape="point"];
p5 [label="",shape="point"];
p6 [label="",shape="point"];
mvreq [label="Robot move request",shape="Mrecord"];
mvrej [label="Reject move",shape="Mrecord"];
retht [label="Return Hint tile",shape="Mrecord",style=filled];
ordex [label="Ordinary exchange",shape="Mrecord"];
spcex [label="Special exchange",shape="Mrecord"];
mvreq->p3;
p3->mvrej;
p3->p4;
p4->retht;
retht->p5;
p5->p6;
p6->ordex;
p6->spcex;
}

	\end{minipage}

	\subsubsection{Return Conveyor tile}
	\begin{minipage}{\linewidth}
		If the move request is valid, and the robot is moved to a conveyor tile, the board again returns \emph{SUCCESS} to the controller, which forwards this message to the robot that was moved. The board will then notify the controller that the robot was moved successfully. The board will make its snapshot, send it to the controller and the controller will notify the viewer; the controller will also forward the \emph{SUCCESS} from the board to the robot. The board will then send a message to the controller that the robot was moved automatically (this was due to the conveyor belt) and the controller will forward this message to the robot, using notifyAutoMovement. As in the 'Return Hint tile' MSC, the notifyAutoMovement from the board to the controller contains the robot that should be notified.

		\begin{msc}
msc
{

d [label="Board"],
c [label="Controller"],
a [label="Robot"],
v [label="Viewer"];

d box d [label=""],
c box c [label=""],
a box a [label=""],
v box v [label=""];

|||;

d rbox d [label="calculateNewLocation(localCoords, robot)"];
d rbox d [label="saveLocation(absCoords, robot)"];
d >> c [label="SUCCESS"];
...;
c >> a [label="SUCCESS"];
c -> v [label="notifyStateChange()"];
c -> a [label="notifyAutoMovement()"];
...;
|||;

a box a [label="", textbgcolor="black"],
c box c [label="", textbgcolor="black"],
d box d [label="", textbgcolor="black"],
v box v [label="", textbgcolor="black"];

}
\end{msc}
\digraph[scale=0.5]{HMSC_mvct}{
rankdir=LR;
p3 [label="",shape="point"];
p4 [label="",shape="point"];
p5 [label="",shape="point"];
p6 [label="",shape="point"];
notrob1 [label="Notify robots",shape="Mrecord"];
p55 [label="",shape="point"];
p66 [label="",shape="point"];
mvreq [label="Robot move request",shape="Mrecord"];
mvrej [label="Reject move",shape="Mrecord"];
retct [label="Return Conveyor tile",shape="Mrecord",style=filled];
ordex [label="Ordinary exchange",shape="Mrecord"];
spcex [label="Special exchange",shape="Mrecord"];
mvreq->p3;
p3->mvrej;
p3->p4;
p4->retct;
retct->p5;
p5->p55;
p55->p66;
p66->p6;
p55->notrob1;
notrob1->p66;
p6->ordex;
p6->spcex;
}

	\end{minipage}

	\subsubsection{Return Home tile}
	\begin{minipage}{\linewidth}
	   If the move request is valid, and the robot is moved to its home tile, the board will notify the controller that the robot that moved wins the game, using the response \emph{WIN}. The controller notifies the viewer that the state of the board has changed, using the board snapshot it received from the board.

		\begin{msc}
msc
{

a [label="Robot type A"],
c [label="Controller"],
d [label="Board"],
r [label="Rule"];

a box a [label=""],
c box c [label=""],
d box d [label=""],
r box r [label=""];

|||;

c => d [label="Make move"];
d rbox d [label="Save location of robot"];
d rbox d [label="Save which robot wins"];
d >> c [label="WIN"];
c -> a [label="WIN"];

|||;

a box a [label="", textbgcolor="black"],
c box c [label="", textbgcolor="black"],
d box d [label="", textbgcolor="black"],
r box r [label="", textbgcolor="black"];

}
\end{msc}

	\end{minipage}

	\subsubsection{Reject move}
	\begin{minipage}{\linewidth}
		If the move request is, for whatever reason, not valid, the board will notify the controller of this, using the response \emph{FAILED}. The controller will forward this to the robot that tried to move.

		\begin{msc}
msc
{

d [label="Board"],
c [label="Controller"],
a [label="Robot"];

d box d [label=""],
c box c [label=""],
a box a [label=""];


|||;

d >> c [label="FAILED"];
c >> a [label="FAILED"];

|||;

a box a [label="", textbgcolor="black"],
c box c [label="", textbgcolor="black"],
d box d [label="", textbgcolor="black"];

}
\end{msc}
\digraph[scale=0.5]{HMSC_mvrej}{
rankdir=LR;
p2 [label="",shape="point"];
p3 [label="",shape="point"];
init [label="Initialize",shape="Mrecord"];
mvreq [label="Robot move request",shape="Mrecord"];
mvrej [label="Reject move",shape="Mrecord",style=filled];
init->p2;
p2->mvreq;
mvreq->p3;
p3->mvrej;
mvrej->p2;
}

	\end{minipage}

	\advance\count17 by -6

	\subsubsection{Ordinary exchange}
	\begin{minipage}{\linewidth}
		The controller requests the board to do a tile exchange. The board will get two valid tiles, swap them and returns an empty RobotPair to signal that there were no robots on the tiles. Note that the 'getValidTiles'-function must maintain the "always reachable home tile"-invariant: each robot must always be able to reach its home tile. The board will calculate this internally, in order to select two valid tiles without robots in the ordinary exchange. Again, the viewer has to be notified that the board has changed because of the tiles exchange.

		\begin{msc}
msc
{

b [label="Board"],
c [label="Controller"],

b box b [label=""],
c box c [label=""],

|||;

c=>b [label="Get two valid tiles"];
b>>c [label="two valid tiles"];
c=>b [label="exchange tiles"];
b rbox b [label="exchange two tiles"];
b->c [label="tiles exchanged"];

|||;

b box b [label="",textbgcolor="black"],
c box c [label="",textbgcolor="black"],

}
\end{msc}

	\end{minipage}

	\subsubsection{Special exchange}
	\begin{minipage}{\linewidth}
		As in the ordinary exchange, the controller requests the board to do a tile exchange. The board will find two valid tiles (in this case, with on at least one of them a robot or a conveyor belt) and swap them. The board will return a RobotPair with the robots that have been selected. Note that this RobotPair will be either an empty RobotPair, a RobotPair with one robot or a RobotPair with two robots. The selected conveyor tiles and robots will be rotated. The controller will notify all players that have been moved and then notify the viewer that the state of the board has changed. Hence, in case of an empty RobotPair, no robots will be notified; if RobotPair contains two robots, both robots will be notified. The viewer is also notified that the board has changed. \\
In the MSC below, one robot and one conveyor belt have been selected. Note that robot2 will not be notified, since it was not part of the tiles exchange. As in the MSC 'Ordinary exchange', the board must make sure that the "always reachable home tile"-invariant is maintained.

		\begin{msc}
msc
{

b [label="Board"],
c [label="Controller"],
p1 [label="Robot1"],
p2 [label="Robot2"],
v [label="Viewer"];

c box c [label=""],
b box b [label=""],
p1 box p1 [label=""],
p2 box p2 [label=""],
v box v [label=""];

|||;
c=>b [label="requestTilesExchange()"];
b rbox b [label="getValidTiles()"];
b rbox b [label="exchange two valid tiles"];
b rbox b [label="rotate robot1 and rotate tiles"];
b>>c [label="RobotPair with one robot"];
...;
c->p1 [label="notifyAutoMovement()"];
c -> v [label="notifyStateChange()"];
...;

|||;

b box b [label="",textbgcolor="black"],
c box c [label="",textbgcolor="black"],
p1 box p1 [label="",textbgcolor="black"],
p2 box p2 [label="",textbgcolor="black"],
v box v [label="",textbgcolor="black"];

}
\end{msc}

\digraph[scale=0.5]{HMSC_exchsp}{
rankdir=LR;
p2 [label="",shape="point"];
p5 [label="",shape="point"];
p6 [label="",shape="point"];
p7 [label="",shape="point"];
p22 [label="",shape="point"];
p77 [label="",shape="point"];
notrob2 [label="Notify robots",shape="Mrecord"];
init [label="Initialize",shape="Mrecord"];
mvreq [label="Robot move request",shape="Mrecord"];
retnt [label="Return Normal tile",shape="Mrecord"];
retht [label="Return Hint tile",shape="Mrecord"];
retct [label="Return Conveyor tile",shape="Mrecord"];
spcex [label="Special exchange",shape="Mrecord",style=filled];
init->p2;
p2->mvreq;
retnt->p5;
retht->p5;
retct->p5;
p5->p6;
p6->spcex;
spcex->p7;
p7->p77;
p77->p22;
p22->p2;
p77->notrob2;
notrob2->p22;
}

	\end{minipage}	

	\subsubsection{Notify robots}
	\begin{minipage}{\linewidth}
		The board sends a notification to the controller, which forwards this to the robot that was moved. This MSC is used when a robot is moved due to the move of another robot, for example when a robot that blocked the end of a conveyor belt moves away and the last robot on the belt should move to the end.

		
\begin{msc}
msc
{

d [label="Board"],
c [label="Controller"],
a [label="Robot"],
v [label="Viewer"];

d box d [label=""],
c box c [label=""],
a box a [label=""],
v box v [label=""];

|||;

d rbox d [label="calculateNewLocation(localCoords, robot)"];
d rbox d [label="saveLocation(absCoords, robot)"];
d >> c [label="SUCCESS"];
c >> a [label="SUCCESS"];
c -> v [label="notifyStateChange()"];
c -> a [label="notifyAutoMovement()"];
|||;

a box a [label="", textbgcolor="black"],
c box c [label="", textbgcolor="black"],
d box d [label="", textbgcolor="black"],
v box v [label="", textbgcolor="black"];

}
\end{msc}

	\end{minipage}

	\subsubsection{End game}
	\begin{minipage}{\linewidth}
		If a robot has reached its home tile, the controller will send a terminate message to all losing robots, which will then terminate safely. The controller will then notify the viewer which robot won the game, so it can show the end game animation. When the animation is done, the viewer detaches itself from the controller, using removeViewer. The controller will then send a message to the board that everything it done and the board can reset and will then terminate. Note that the winning robot can terminate on its own, since its termination does not affect the other components of the game; the viewer can show an appropriate animation that reflects which robot has won. The board can then choose to either reset to start a new game or terminate; since this is a non-deterministic choice from the board, this is not modeled in the MSC below.

		\begin{msc}
msc
{

b [label="Board"],
c [label="Controller"],
p [label="Robot Lost"],
pw [label="Robot Win"],
v [label="View"];

b box b [label=""],
c box c [label=""],
p box p [label=""],
pw box pw [label=""],
v box v [label=""];

|||;

c -> p [label="terminate"];
p rbox p [label="terminate()"];
c note p [label="All robots who lost the game receive a terminate from the controller."];
c -> v [label="notifyGameOver(robot)"];
v rbox v [label="fireworks"];
v -> c [label="removeViewer()"];
c -> b [label="canReset"];
v rbox v [label="terminate"],
b rbox b [label="reset()"],
c rbox c [label="terminate()"],
pw rbox pw [label="terminate()"];


|||;

b box b [label="",textbgcolor="black"],
c box c [label="",textbgcolor="black"],
p box p [label="",textbgcolor="black"],
v box v [label="",textbgcolor="black"];

}
\end{msc}
\digraph[scale=0.5]{HMSC_end}{
rankdir=LR;
begin [label="",shape="invtriangle"];
end [label="",shape="triangle"];
p1 [label="",shape="point"];
p8 [label="",shape="point"];
init [label="Initialize",shape="Mrecord"];
retmt [label="Return Home tile",shape="Mrecord"];
endgame [label="End game",shape="Mrecord",style=filled];
begin->p1;
p1->init;
retmt->endgame;
endgame->p8;
p8->p1;
p8->end;
}

	\end{minipage}
