\subsection{Class diagram}
	The following graphic is the class diagram that we will use for the specification and the implementation.

	%\includegraphics[width=\linewidth]{classdiagram.pdf}
	\let\l=\relax
\let\<=\relax
\let\>=\relax
\digraph[scale=.3]{classdiagram}{
margin=0
fontsize=8
fontname=Helvetica
compound=true
splines=ortho
node [fontsize=8, fontname=Helvetica, shape=record]
edge [fontsize=8, fontname=Helvetica, arrowhead=open, labeldistance=2]
Board [label="{Board||}"]
Hint [label="{[Hint]||}"]
BoardResponse [label="{[BoardResponse]||}"]
Viewer [label="{Viewer||}"]
BoardSnapshot [label="{BoardSnapshot||}"]
Controller [label="{Controller||}"]
AbsoluteCoord [label="{AbsoluteCoord||}"]
Tile [label="{Tile||}"]
/**/
subgraph cluster_Tiles {
NormalTile [label="{NormalTile||}"]
HomeTile [label="{HomeTile||}"]
HintTile [label="{HintTile||}"]
ConveyorTile [label="{ConveyorTile||}"]
BrokenRobotTile [label="{BrokenRobotTile||}"]
}
/**/
Robot [label="{Robot||}"]
RelativeCoord [label="{RelativeCoord||}"]
Rule [label="{\<\<Rule\>\>||}"]
Rotation [label="{[Rotation]||}"]
/**/
Board->Controller [taillabel=1, headlabel="0..*"]
Board->Tile [arrowtail=diamond,dir=both, taillabel=1,headlabel="*"]
Board->Robot [taillabel=1, headlabel="0..*"]
/**/
Controller->Viewer [taillabel=1, headlabel=1, arrowhead=none]
Controller->Robot [taillabel=1, headlabel="*", arrowhead=none]
/**/
Tile->Robot [taillabel=1, headlabel="              0..1"]
/**/
HomeTile->Robot [taillabel=1, headlabel="              1"]
/**/
Robot->Rule [taillabel="*", headlabel=1]
/**/
BoardSnapshot->Tile [taillabel=1, headlabel="*"]
/**/
NormalTile->Tile [ltail=cluster_Tiles,arrowhead=empty]
/**/
Board->Hint [style=dashed]
Board->BoardResponse [style=dashed]
Board->AbsoluteCoord [style=dashed]
Viewer->BoardSnapshot [style=dashed]
Controller->AbsoluteCoord [style=dashed]
Robot->RelativeCoord [style=dashed]
Rule->Rotation [style=dashed]
ConveyorTile->Rotation [style=dashed]
} 


\subsection{Class description}
Abstract classes are indicated by guillemets and enumerators by brackets.
	\begin{description}
        \item[Hint] An enumeration that contains all possible hints that a Robot can receive from a hint tile.
        \item[BoardResponse] An enumerator that contains all the possible responses that the board can give the controller when it makes a move request.
		\item[Board] The board as it was given in the informal specification.
		\item[Controller] The main controller as it was given in the informal specification.
        \item[Viewer] The viewer as it was given in the informal specification.
        \item[AbsoluteCoord] A data class that contains the x and y coordinate of an absolute coordinate.
		\item[BoardSnapshot] A data class that contains a snapshot of the board, i.e. a copy of all the tiles in the board.
		\item[Tile] Used to model the tiles that the Board consists of.
		\item[NormalTile] Tiles without a special meaning (specialization of the Tile class).
		\item[HomeTile] Tiles that are the \"home\" of each robot (specialization of the Tile class).
		\item[HintTile] Tiles that return a hint as to where the robot\'s home is (specialization of the Tile class).
		\item[ConveyorTile] Tiles that change the position and rotation of robots (specialization of the Tile class).
		\item[BrokenRobotTile] Tiles that are occupied by a defective robot (specialization of the Tile class).
		\item[Robot] This is used for both Robot A and Robot B in the informal specification.
		\item[Rule] An abstract class that is used to model the behaviour of Robot A and B in the Robot class. Any class that defines a rule inherits from this class.
		\item[RelativeCoord] A data class that contains the x and y coordinate of a relative coordinate.
		\item[Rotation] An enumeration that is used to model the rotation in robots and conveyors.
	\end{description}
